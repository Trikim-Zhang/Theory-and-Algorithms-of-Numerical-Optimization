\section{无约束优化}
\subsection{最优性条件的应用}
\begin{theorem}
    设$\boldsymbol{x}^*$为优化问题$\min\limits_x\in\mathbb{R}^nf(\boldsymbol{x})$ 的最优解,则 $\nabla f(\boldsymbol{x}^*)=\mathbf{0},\nabla^2f(\boldsymbol{x}^*)$半正定。
\end{theorem}
\begin{theorem}
    对优化问题$\min\limits_{\boldsymbol{x}\in\mathbb{R}^n}f(\boldsymbol{x})$ ,设$\nabla f(\boldsymbol{x}^*)=0,\nabla^2f(\boldsymbol{x}^*)$正定,
则$\boldsymbol{x}^*$是该优化问题的严格最优解,
\end{theorem}
\begin{example}
    优化问题的解析解\Stars{5}
    \[
        \min\limits_{x>0,y\geqslant 0}f(x,y)=\dfrac{10}{x}+\dfrac{(x-y)^{2}}{2x}+\dfrac{3y^{2}}{2x}
    \]
    \begin{solution}
        先忽略约束:利用$\min\limits_{x,y}f(x,y)=\min\limits_x\min\limits_yf(x,y)$,先固定$x$,关于$y$做内层优化,再求解关于$x$的外层优化
        \[
            \min\limits_{y}f(x,y)=\frac{10}{x}+\frac{(x-y)^{2}}{2x}+\frac{3y^{2}}{2x}
        \]
        目标函数关于$y$为凸函数,利用最优性条件得最优解
        \[
            \begin{array}{c}
                f(x,y)=\dfrac{1}{2x}\left( 20+x^2-2xy+4y^2 \right)\\
                y = \dfrac{1}{4}x
            \end{array}
        \]
        将上述最优解代入目标函数得外层优化问题
        \[
            \min f(x)=\dfrac{10}{x}+\dfrac{3}{8}x
        \]
        目标函数关于$x$为凸函数(二阶导数大于0)。再利用最优性条件得$x=\dfrac43\sqrt{15},y=\dfrac{1}{4}x=\dfrac{1}{3}\sqrt{15}$

        它们满足约束条件,自然为原问题的最优解。
    \end{solution}
\end{example}
\subsection{最速下降算法}
\begin{example}
    利用最速下降方法求\Stars{5}
    \[
        \min\limits_{\boldsymbol{x}\in\mathbb{R}^2}f(x_1,x_2) = \dfrac{1}{3}x_{1}^2+\dfrac{1}{2}x_2^2
    \]
    \begin{solution}
        显然,唯一最优解$\boldsymbol{x}^* = \left( 0,0 \right)$,取初始点$\boldsymbol{x}_0 = (3,2)$,那么最速下降方法产生的迭代点列
        $\boldsymbol{x}_k$

        \begin{itemize}
            \item $k = 0$时
            \[
                \boldsymbol{g}_0 = \nabla f(\boldsymbol{x}^{(0)})  =  \begin{pmatrix}
                    2\\2
                \end{pmatrix}
            \]
            \[
                \begin{aligned}
                    \alpha_0 &= \arg\min_{\alpha>0}f(\boldsymbol{x}^{(0)}-\alpha\boldsymbol{g}^{(0)})\\
                    &=\arg\min_{\alpha>0}\dfrac{1}{3}\left( 10\alpha^2-24\alpha+\cdots \right) = \dfrac{6}{5}
                \end{aligned}
            \]
            \[
                \boldsymbol{x}^{(1)} = \boldsymbol{x}^{(0)}-\alpha\boldsymbol{g}^{(0)  } = (3,2)-\dfrac{6}{5}(2,2)  = (\dfrac{3}{5},-\dfrac{2}{5})
            \]
            \item $k = 1,\cdots $
        \end{itemize}
        综上,得到迭代序列$\boldsymbol{x}^{(k)}$
        \[
            \boldsymbol{x}_{k}=\left(\frac{3}{5^{k}};(-1)^{k}\frac{2}{5^{k}}\right)\overset{k\to\infty}{\longrightarrow}(0;0)
        \]
        全局收敛,收敛速度线性!
        有$\|\boldsymbol{x}_k-\boldsymbol{x}^*\|=\sqrt{13}\big(\frac15\big)^k.$那么
        \[
            \begin{aligned}
                \frac{\|\boldsymbol{x}_{k+1}-\boldsymbol{x}^*\|}{\|\boldsymbol{x}_k-\boldsymbol{x}^*\|} & =\frac15<\frac15\sqrt{\frac32}=\frac{\lambda_1-\lambda_2}{\lambda_1+\lambda_2}\sqrt{\frac{\lambda_1}{\lambda_2}}\\
                \frac{f(\boldsymbol{x}_{k+1})-f(\boldsymbol{x}^*)}{f(\boldsymbol{x}_k)-f(\boldsymbol{x}^*)} & =\frac1{25}=\left(\frac{\lambda_1-\lambda_2}{\lambda_1+\lambda_2}\right)^2
            \end{aligned}
        \]
    \end{solution}
\end{example}
\subsection{牛顿算法}
\begin{example}
    用牛顿法求解\Stars{5}
    \[
        \min f(x)=\sqrt{1+x^2}
    \]
    \begin{solution}
        $0$为最优解
    
        目标函数导数$f'(x)=\dfrac{x}{\sqrt{1+x^2}}\quad f''(x)=\dfrac{1}{(1+x^2)^{3/2}}$
        
        迭代过程
        \[
            \begin{aligned}
                x_{k+1}& =x_{k}-\frac{f'(x_{k})}{f''(x_{k})}  \\
                &=x_{k}-x_{k}(1+x_{k}^{2}) \\
                &=-x_{k}^{3}
            \end{aligned}
        \]
        \begin{itemize}
            \item 当$x_0<1$,算法快速收敛到最优解;
            \item 当$x_0\geqslant 1$,算法不收敛
        \end{itemize}
    \end{solution}
\end{example}

\begin{example}
    用牛顿算法求解\Stars{5}{}
    \[
        \min f(\boldsymbol{x}) = 4x_1^2+x_2^2-x_1^2x_2
    \]
    \begin{solution}
        最优解$\boldsymbol{x}^* = (0,0)$

        函数鞍$(2\sqrt{2},4),\,(-2\sqrt{2},4)$

        目标函数梯度信息
        \[
            \nabla f(\boldsymbol{x}) = \begin{pmatrix}
                8x_1-2x_1x_2\\2x_2-x_1^2
            \end{pmatrix},\quad
            \nabla^2 f(\boldsymbol{x}) = \begin{pmatrix}
                8-2x_2 & -2x_1 \\
                -2-x_1 & 2 
            \end{pmatrix},
        \]
        取精度$\varepsilon = 10^{-3}$和不同初始点
    \end{solution}
\end{example}
\subsection{线性共轭梯度法}
\begin{example}
    利用共轭梯度法求$\boldsymbol{Ax} = \boldsymbol{b} = \boldsymbol{0}$或者说,利用共轭梯度法求\quad\Stars{5}
    \[
        \min x_1^2+\dfrac{1}{2}x_2^2+\dfrac{1}{2}x_3^2
    \]
    其中,
    \[
        \boldsymbol{A} = \begin{bmatrix}
            1 & 0 & 0\\
            0 & \frac{1}{2} & 0 \\
            0 & 0 & \frac{1}{2}
        \end{bmatrix},\quad
        \boldsymbol{b} = \begin{pmatrix}
            0\\0\\0
        \end{pmatrix}
    \]

    \begin{solution}
        取初始点$\boldsymbol{x}_0 = (1,1,1)^{\mathrm{T}} $,迭代过程:
        \begin{enumerate}
            \item $\boldsymbol{x}_0 = (1,1,1)^{\mathrm{T}} $,$\boldsymbol{g}_0 = \boldsymbol{Ax}_0-\boldsymbol{0} = (2,1,1)^{\mathrm{T}}$,$\beta_{-1} = 0$,$\boldsymbol{d}_{0} = -\boldsymbol{g}_0$.
            \[
                \begin{array}{ll}
                    \alpha &= \arg\min f(\boldsymbol{x}_0+\alpha \boldsymbol{d}_0)\\
                    & = (1-2\alpha)^2+\frac{1}{2}(1-\alpha)^2+\frac{1}{2}(1-\alpha)^2 = \frac{3}{5}
                \end{array}
            \]
            \item $\boldsymbol{x}_1 = \boldsymbol{x}_0+\alpha_0\boldsymbol{d}_0=\frac{1}{5}(-1,2,2)^{\mathrm{T}} $,$\boldsymbol{g}_1 = \boldsymbol{Ax}_1-\boldsymbol{0} = \frac{1}{5}(-1,2,2)^{\mathrm{T}}$,$\beta_{0} = \frac{\boldsymbol{g}_1^{\mathrm{T}}\boldsymbol{g}_1}{\boldsymbol{g}_0^{\mathrm{T}}\boldsymbol{g}_0}= \frac{2}{25}$ 
            \[
                \boldsymbol{d}_{1} = -\boldsymbol{g}_1+\beta_{0}\boldsymbol{d}_0 = -\frac{6}{25}(1,-2,-2)^{\mathrm{T}}
            \]
            \[
                \begin{array}{ll}
                    \alpha &= \arg\min f(\boldsymbol{x}_0+\alpha \boldsymbol{d}_0)\\
                    & = (1-2\alpha)^2+\frac{1}{2}(1-\alpha)^2+\frac{1}{2}(1-\alpha)^2 = \frac{3}{5}
                \end{array}
            \]
            \item $\boldsymbol{x}_2 = \boldsymbol{x}_1 + \alpha_1\boldsymbol{d}_1 = \boldsymbol{0}$,$\|\boldsymbol{g}_2\| = 0$,终止。
        \end{enumerate}

        \begin{table}[htbp]
            \centering
            \begin{tabular}{c|c|c|c|c|c}
                \hline
                $k$ & $\boldsymbol{x}_k$ & $\boldsymbol{g}_k$ & $\beta_{k-1}$ & $\boldsymbol{d}_k$ & $\alpha_k$\\\hline
                $0$ & $ (1,1,1)^{\mathrm{T}} $ & $(2,1,1)^{\mathrm{T}}$ & $0$ & $ -(2,1,1)^{\mathrm{T}} $ & $\frac{3}{5}$\\\hline
                $1$ & $ \frac{1}{5}(-1,2,2)^{\mathrm{T}} $ & $\frac{1}{5}(-2,2,2)^{\mathrm{T}}$ & $\frac{2}{25}$ & $ -\frac{6}{25}(1,-2,-2)^{\mathrm{T}} $ & $\frac{5}{6}$\\\hline
                $2$ & $(0,0,0)^{\mathrm{T}}$ & $(0,0,0)^{\mathrm{T}}$ & \\\hline
            \end{tabular}
        \end{table}
    \end{solution}
\end{example}
