\section{约束优化最优性条件}
\subsection{单约束优化}
\begin{example}
    设$\boldsymbol{a}_1,\boldsymbol{a}_2,\cdots,\boldsymbol{a}_m\in\mathbb{R}^n$满足$\sum\limits_{i = 1}^{m}\boldsymbol{a}_{i}\neq \boldsymbol{0}$求解优化问题\Stars{5}{}
    \[
        \min\{\sum_{i=1}^{m}\|\boldsymbol{x}-\boldsymbol{a}_{i}\|^{2}\mid\boldsymbol{x}^{\mathrm{T}}\boldsymbol{x}=1\}
    \]
    \begin{solution}
        令
        \[
            \begin{aligned}
                L(\boldsymbol{x},\lambda)&=\sum\limits_{i = 1}^{m}\|\boldsymbol{x}-\boldsymbol{a}_{i}\|^{2}-\lambda(\boldsymbol{x}^{\mathrm{T}}\boldsymbol{x}-1)\\
                &=\sum_{i=1}^{m}\sum_{j=1}^{n}\left( \boldsymbol{x}_j-(\boldsymbol{a}_{i})_{j} \right)^2-\lambda\left( \sum_{j = 1}^{n}\boldsymbol{x}_j^2-1\right)
            \end{aligned}
        \]
        对$x_k$求偏导,得到
        \[
            \dfrac{\partial L}{\partial x_{k}} = \sum_{i = 1}^{m}2(x_{k}-(\boldsymbol{a}_{i})_{k})-\lambda\cdot 2 x_{k}
        \]
        故而有
        \[
            \nabla_{x}L = \begin{pmatrix}
                \sum_{i = 1}^{m}2(x_{1}-(\boldsymbol{a}_{i})_{1})-\lambda 2x_{1}\\
                \sum_{i = 1}^{m}2(x_{2}-(\boldsymbol{a}_{i})_{2})-\lambda 2x_{2}\\
                \vdots\\
                \sum_{i = 1}^{m}2(x_{n}-(\boldsymbol{a}_{i})_{n})-\lambda 2x_{n}\\
            \end{pmatrix} 
        \]
        所以,有
        \[
            \begin{cases}
                \sum\limits_{i = 1}^{m}2(\boldsymbol{x-\boldsymbol{a}_{i}}) = 2\lambda\boldsymbol{x}\\
                \boldsymbol{x}^{\mathrm{T}}\boldsymbol{x}=1
            \end{cases}
        \]
        设方程的解为$\boldsymbol{x}^*$,则$\boldsymbol{x}^*$与$\sum\limits_{i = 1}^{m}\boldsymbol{a}_i$同向或者反向。

        若$\sum\limits_{i = 1}^{m}\boldsymbol{a}_i \neq \boldsymbol{0}$,则
        \[
            \boldsymbol{x}^*=\dfrac{\sum_{i=1}^m\boldsymbol{a}_i}{\|\sum_{i=1}^m\boldsymbol{a}_i\|}
        \]
        相应地,最优值为
        \[
            \begin{aligned}
                f(\boldsymbol{x}^*) &= \sum_{i=1}^{m}\|\boldsymbol{x}^*-\boldsymbol{a}_{i}\|^{2} \\
                & = \sum_{i=1}^{m}\left< \boldsymbol{x}^*-\boldsymbol{a}_{i},\boldsymbol{x}^*-\boldsymbol{a}_{i} \right>\\
                & =\sum_{i=1}^{m}\|\boldsymbol{x}^*\|^2-2\left<\dfrac{\sum_{i=1}^m\boldsymbol{a}_i}{{\|\sum_{i=1}^m\boldsymbol{a}_i\|}},\sum_{i=1}^m\boldsymbol{a}_i\right>+\sum_{i=1}^m\|\boldsymbol{a}_i\|^2\\
                &=m-2\|\sum_{i=1}^m\boldsymbol{a}_i\|+\sum_{i=1}^m\|\boldsymbol{a}_i\|^2
            \end{aligned}
        \]
    \end{solution}
\end{example}
\subsection{线性规划的对偶}
\begin{example}
    线性规划的对偶\Stars{5}{}
    \[
        \begin{array}{rl}
            \min & \boldsymbol{c}^\mathrm{T}\boldsymbol{x}\\
            \mathrm{s.t.} & \boldsymbol{Ax}=\boldsymbol{b}\\
            & \boldsymbol{x}\geqslant\boldsymbol{0}
        \end{array}
    \]
    对偶
    \[
        \max_{\boldsymbol{u}\geq0,\boldsymbol{v}}\min_{\boldsymbol{x}\in \mathbb{R}^n}L(\boldsymbol{x},\boldsymbol{u},\boldsymbol{v})
    \]
    \[
        \begin{aligned}
            L(\boldsymbol{x};\boldsymbol{u},\boldsymbol{v})&=\boldsymbol{c}^\mathrm{T}\boldsymbol{x}-\boldsymbol{u}^\mathrm{T}\boldsymbol{x}-\boldsymbol{v}^\mathrm{T}(\boldsymbol{A}\boldsymbol{x}-\boldsymbol{b})\\
            &=\left(\boldsymbol{c}-\boldsymbol{u}-\boldsymbol{A}^\mathrm{T}\boldsymbol{v}\right)^\mathrm{T}\boldsymbol{x}+\boldsymbol{v}^\mathrm{T}\boldsymbol{b},\quad\boldsymbol{u}\geqslant\boldsymbol{0}.
        \end{aligned}
    \]
    \[
        \begin{aligned}
            \min_{\boldsymbol{x}\in\mathbb{R}^n}L(\boldsymbol{x};\boldsymbol{u},\boldsymbol{v})&=\min_{\boldsymbol{x}\in\mathbb{R}^n}\left(\boldsymbol{c}-\boldsymbol{u}-\boldsymbol{A}^\mathrm{T}\boldsymbol{v}\right)^\mathrm{T}\boldsymbol{x}+\boldsymbol{v}^\mathrm{T}\boldsymbol{b}\\
            &=\begin{cases}
                -\infty, & \boldsymbol{c}-\boldsymbol{u}-\boldsymbol{A}^\mathrm{T}\boldsymbol{v}\neq \boldsymbol{0} ,\\
                \boldsymbol{v}^\mathrm{T}\boldsymbol{b},& \boldsymbol{c}-\boldsymbol{u}-\boldsymbol{A}^\mathrm{T}\boldsymbol{v}= \boldsymbol{0},
            \end{cases}
        \end{aligned}
    \]
    \[
        \begin{array}{rl}
            \max & (\boldsymbol{c}-\boldsymbol{u}-\boldsymbol{A}^\mathrm{T}\boldsymbol{v})^\mathrm{T}\boldsymbol{x}+\boldsymbol{b}^\mathrm{T}\boldsymbol{v}  \\
            \mathrm{s.t.} & \nabla_{x}L(\boldsymbol{x};\boldsymbol{u},\boldsymbol{v})=\boldsymbol{c}-\boldsymbol{u}-\boldsymbol{A}^{\mathrm{T}}\boldsymbol{v}=\boldsymbol{0}  \\
            &\boldsymbol{u}\geqslant 0
        \end{array}
    \]
    化简得到
    \[
        \begin{array}{rl}
            \max&\boldsymbol{v}^\mathrm{T}\boldsymbol{b}\\
            \mathrm{s.t.}&\boldsymbol{A}^\mathrm{T}\boldsymbol{v}\leqslant \boldsymbol{c}
        \end{array}
    \]
\end{example}
\subsection{混合约束优化的对偶}
\begin{example}
    求下述优化问题的对偶\Stars{5}{}
    \[
        \begin{array}{rl}
            \operatorname*{min}&f(x)=x_{1}^{2}+x_{2}^{2}\\
            \mathrm{s.t.} & x_{1}+x_{2}-4\geq 0\\
            &x\geq 0
        \end{array}    
    \]
    \begin{solution}
        Lagrange函数
        \[
            L(x,\lambda)=x_{1}^{2}+x_{2}^{2}-\lambda(x_{1}+x_{2}-4),\quad x\geq 0,\quad\lambda\geq 0
        \]
        Lagrange对偶$\max\limits_{\lambda\geqslant 0}\min\limits_{x\geqslant 0}L(\lambda,\boldsymbol{x})$
        \[
            \begin{aligned}
                \theta(\lambda)& =\min_{x\geq0}L(x,\lambda)  \\
                &=\min_{x\geq0}\{x_{1}^{2}+x_{2}^{2}-\lambda(x_{1}+x_{2}-4)\} \\
                &=\left(\operatorname*{min}_{x_{1}\geq0}\{x_{1}^{2}-\lambda x_{1}\}+\operatorname*{min}_{x_{2}\geq0}\{x_{2}^{2}-\lambda x_{2}\}+4\lambda\right) \\
                &=-\frac14\lambda^{2}-\frac14\lambda^{2}+4\lambda  \\
                &=-\frac{1}{2}\lambda^{2}+4\lambda 
            \end{aligned}  
        \]
        其中$x_1^* = x_2^* = \dfrac{1}{2}\lambda$
        \[
            \max_{\lambda\geq 0}\theta(\lambda)=-\frac{1}{2}\lambda^{2}+4\lambda=8
        \]
    \end{solution}    
\end{example}
\begin{example}
    求解无约束优化问题\Stars{5}{}
    \[
        \min_{\boldsymbol{x}\in\mathbb{R}^n}\frac12\|\boldsymbol{A}\boldsymbol{x}-\boldsymbol{b}\|^2+\lambda\|\boldsymbol{x}\|_1
    \]
    \begin{solution}
        令$\boldsymbol{y} = \boldsymbol{x}$得
        \[
            \begin{array}{rl}
                \min&\dfrac{1}{2}\|\boldsymbol{A}\boldsymbol{x}-\boldsymbol{b}\|^2+\lambda\|\boldsymbol{y}\|_1\\
                \mathrm{s.t.}&\boldsymbol{y}-\boldsymbol{x}=0
            \end{array}
        \]
        对偶
        \[
            \max_{\boldsymbol{z}\in\mathbb{R}^n}\min_{\boldsymbol{x},\boldsymbol{y}\in\mathbb{R}^n} =\frac12\|\boldsymbol{A}\boldsymbol{x}-\boldsymbol{b}\|^2+\lambda\|\boldsymbol{y}\|_1+\boldsymbol{z}^{\mathrm{T}}(\boldsymbol{x}-\boldsymbol{y}).
        \]
        变量可分离
        \[
            \boxed{\min_{\boldsymbol{x}\in\mathbb{R}^n}\left(\frac12\|\boldsymbol{Ax}-\boldsymbol{b}\|^2+\boldsymbol{z}^{\mathrm{T}} \boldsymbol{x}\right)} + \boxed{\min_{\boldsymbol{y}\in\mathbb{R}^n}\left(\lambda\|\boldsymbol{y}\|_1-\boldsymbol{z}^{\mathrm{T}} \boldsymbol{y}\right)}
        \]
        对于$\min\limits_{\boldsymbol{y}\in\mathbb{R}^n}\left(\lambda\|\boldsymbol{y}\|_1-\boldsymbol{z}^{\mathrm{T}} \boldsymbol{y}\right)$
        有
        \[
            \min_{\boldsymbol{y}\in\mathbb{R}^n}\lambda\|\boldsymbol{y}\|_1-\boldsymbol{z}^{\mathrm{T}}\boldsymbol{y}=
            \begin{cases}
                0,& \|\boldsymbol{z}\|_\infty\leqslant\lambda,\\
                -\infty,& \|\boldsymbol{z}\|_\infty>\lambda.
            \end{cases}
        \]
        综上,将$\boldsymbol{x}^* = \left( \boldsymbol{A}^{\mathrm{T}}\boldsymbol{A} \right)^{-1}(\boldsymbol{A}^{\mathrm{T}}\boldsymbol{b}-\boldsymbol{z})$代入得到
        \[
            \begin{aligned}
                &\max_{\boldsymbol{z}\in\mathbb{R}^n}\min_{\boldsymbol{x},\boldsymbol{y}\in\mathbb{R}^n}L(\boldsymbol{x},\boldsymbol{y},\boldsymbol{z})\\
                &=\max_{\|\boldsymbol{z}\|_\infty\leqslant\lambda}\frac12\|\boldsymbol{A}(\boldsymbol{A}^{\mathrm{T}}\boldsymbol{A})^{-1}(\boldsymbol{A}^{\mathrm{T}}\boldsymbol{b}-\boldsymbol{z})-\boldsymbol{b}\|^2+\boldsymbol{z}^{\mathrm{T}}(\boldsymbol{A}^{\mathrm{T}}\boldsymbol{A})^{-1}(\boldsymbol{A}^{\mathrm{T}}\boldsymbol{b}-\boldsymbol{z})\\
                &=\max_{\|\boldsymbol{z}\|_\infty\leqslant\lambda}-\frac12\boldsymbol{z}^{\mathrm{T}}(\boldsymbol{A}^{\mathrm{T}}\boldsymbol{A})^{-1}\boldsymbol{z}+\boldsymbol{z}^{\mathrm{T}}(\boldsymbol{A}^{\mathrm{T}}\boldsymbol{A})^{-1}\boldsymbol{A}^{\mathrm{T}}\boldsymbol{b}\\
                &=\min_{\|\boldsymbol{z}\|_\infty\leqslant\lambda}\frac12\boldsymbol{z}^{\mathrm{T}}(\boldsymbol{A}^{\mathrm{T}}\boldsymbol{A})^{-1}\boldsymbol{z}-\boldsymbol{z}^{\mathrm{T}}(\boldsymbol{A}^{\mathrm{T}}\boldsymbol{A})^{-1}\boldsymbol{A}^{\mathrm{T}}\boldsymbol{b}
            \end{aligned}
        \]
        \begin{itemize}
            \item 原问题为线性约束的凸优化问题, 强对偶定理成立.
            \item 从而,原问题等价地化为一带简单约束的光滑优化问题.
        \end{itemize}
    \end{solution}
\end{example}
\subsection{等式约束二次规划—消去法}
\subsection{等式约束二次规划--Lagrange方法}
\begin{example}
    求解以下\colorbox{cyan!50}{无约束二次规划}\Stars{4}{}
    \[
        \begin{array}{rl}
            \min & Q (\boldsymbol{x}) = x_1^2-x_2^2-x_3^2\\
            \operatorname{s.t.} & x_1+x_2+x_3 = 1\\
            & x_2-x_3 = 1
        \end{array}
    \]
    \begin{solution}
        \[
            \boldsymbol{x}_B = \begin{pmatrix}
                x_1\\x_2
            \end{pmatrix},\, \boldsymbol{x}_N = x_3
        \]
        则$x_2 = x_3+1,x_1 = -2x_3$
        问题转化为
        \[
            \begin{array}{l}
                \min Q(x_3) = 4x_3^2-(x_3+1)^2-x_3^2\\
                =2x_3^2-x_3-1
            \end{array}
        \]
        得到
        \[
            \begin{array}{cc}
                x_3^* = \dfrac{1}{2} &
                \boldsymbol{x}^* = \begin{pmatrix}
                    -1\\ 3/2\\ 1/2
                \end{pmatrix}
            \end{array}
        \]
    \end{solution}
\end{example}

\begin{example}
    用Lagrange法求解如下问题\Stars{5}{}
    \[
        \begin{array}{rl}
            \operatorname*{min}& Q(\boldsymbol{x})=3x_{1}^{2}+2x_{1}x_{2}+x_{1}x_{3}+2.5x_{2}^{2}+2x_{2}x_{3}+2x_{3}^{2}-8x_{1}-3x_{2}-3x_{3}\\
            \mathrm{s.t.}&x_{1}+x_{3}=3\\
            &x_{2}+x_{3}=0
        \end{array}
    \]
    \begin{solution}
        由目标函数及约束条件
        \[
            \boldsymbol{G}=
                \begin{pmatrix}
                    6&2&1\\2&5&2\\1&2&4
                \end{pmatrix},\,
            \boldsymbol{g}=
                \begin{pmatrix}
                    -8\\-3\\-3
                \end{pmatrix},\,
            \boldsymbol{A}=
                \begin{pmatrix}
                    1&0&1\\0&1&1
                \end{pmatrix},\,
            \boldsymbol{b}=
                \begin{pmatrix}
                    3\\0
                \end{pmatrix}
            \]
            KKT系统
            \[
                \begin{pmatrix}
                    6&2&1&-1&0\\2&5&2&0&-1\\1&2&4&-1&-1\\1&0&1&0&0\\0&1&1&0&0
                \end{pmatrix}
                \begin{pmatrix}
                    x_1\\x_2\\x_3\\\lambda_1\\\lambda_2
                \end{pmatrix} = 
                \begin{pmatrix}
                    8\\3\\3\\-3\\0
                \end{pmatrix}
            \]
            解得
            \[
                \boldsymbol{x}^{*}=
                \begin{pmatrix}
                    {2}\\{-1}\\{1}
                \end{pmatrix}
                ,\quad\boldsymbol{\lambda}^{*}=
                \begin{pmatrix}
                    {3}\\{-2}
                \end{pmatrix}
            \]
    \end{solution}
\end{example}
\subsection{二次规划有效集方法}
\begin{example}
    求解二次规划问题\Stars{5}{}
    \[
        \begin{array}{rl}
            \operatorname*{min} & Q(\boldsymbol{x})=x_{1}^{2}+x_{2}^{2}-2x_{1}-4x_{2}\\
            \mathrm{s.t.}&-x_{1}-x_{2}+1\geqslant0\\
            &x_{1},x_{2}\geqslant 0
        \end{array}    
    \]
    \begin{solution}
        \colorbox{cyan!50}{$k = 0$:}

        取初始点
        \[
            \boldsymbol{x}^{(0)}=\begin{pmatrix}0\\0\end{pmatrix},\, S_0=\mathcal{A}(\boldsymbol{x}^{(0)})=\{2,3\}
        \]
        求解等式约束优化子问题
        \[
            \begin{array}{rl}
                \min & d_{1}^{2}+d_{2}^{2}-2d_{1}-4d_{2}\\
                \mathrm{s.t.} & d_{1}=0\\
                & d_{2}=0
            \end{array}
        \]
        KKT条件
        \[
            \left\{
                \begin{array}{l}
                    2d_1-2-\lambda_1 = 0\\
                    2d_2-4-\lambda_2 = 0\\
                    d_1 = 0\\
                    d_2 = 0
                \end{array}
            \right.
        \]
        得最优解和相应的Lagrange乘子
        \[
            \boldsymbol{d}^{(0)}=\begin{pmatrix}0\\0\end{pmatrix},\,
            \boldsymbol{\lambda}^{(0)}=\begin{pmatrix}-2\\-4\end{pmatrix}
        \]
        因为$\boldsymbol{d}^{(0)} = \boldsymbol{0}$,新的迭代点为
        \[
            \boldsymbol{x}^{(1)}= \boldsymbol{x}^{(0)} = \begin{pmatrix}0\\0\end{pmatrix}
        \]
        因为$\lambda_{i} <0$,修正指标集
        \[
            S_1=S_0/\{3\}=\mathcal{A}_0/\{3\}=\{2\}
        \]

        \colorbox{cyan!50}{$k = 1$:}

        求解子问题
        \[
            \begin{array}{rl}
                \operatorname*{min}&d_{1}^{2}+d_{2}^{2}-2d_{1}-4d_{2}\\
                \mathrm{s.t.}&d_{1}=0
            \end{array}
        \]
        KKT条件
        \[
            \left\{
                \begin{array}{l}
                    2d_1-2-\lambda_1 = 0\\
                    2d_2-4 = 0\\
                    d_1 = 0\\
                \end{array}
            \right.
        \]
        得最优解
        \[
            \boldsymbol{d}^{(1)}=\begin{pmatrix}0\\2\end{pmatrix},\,
            \boldsymbol{\lambda}^{(1)}=\begin{pmatrix}-2\end{pmatrix}
        \]  
        因为$\boldsymbol{d}^{(1)}\neq \boldsymbol{0}$,转第三步,计算步长$\alpha$,其中$i\in\mathcal{I}/S_{1} = \left\{ 1,2,3 \right\}/\left\{ 2 \right\} = \left\{ 1,3 \right\}$
        \newline
        因为
        \[
            \begin{array}{l}
                \boldsymbol{a}_{1}^{\mathrm{T}}\boldsymbol{d}^{(1)} =
                \begin{pmatrix}
                    -1 & -1
                \end{pmatrix}\cdot
                \begin{pmatrix}
                    0 \\ 2
                \end{pmatrix} = -2 \\
                \boldsymbol{a}_{2}^{\mathrm{T}}\boldsymbol{d}^{(1)}  =
                \begin{pmatrix}
                    0 & 1
                \end{pmatrix}\cdot
                \begin{pmatrix}
                    0 \\ 2
                \end{pmatrix} = 2>0
            \end{array}
        \]
        \[
            \begin{aligned}
                \alpha_{1}& =\min\left\{1,\frac{b_{i}-\boldsymbol{a}_{i}^{\mathrm{T}}\boldsymbol{x}^{(1)}}{\boldsymbol{a}_{i}^{\mathrm{T}}\boldsymbol{d}^{(1)}}\mid i=1,3,\boldsymbol{a}_{i}^{\mathrm{T}}\boldsymbol{d}^{(1)}<0\right\}  \\
                &=\min\left\{1,\dfrac{-1-\begin{pmatrix}
                    -1 & -1
                \end{pmatrix}\cdot
                \begin{pmatrix}
                    0 \\ 0
                \end{pmatrix}}{-2}\right\} = \dfrac{1}{2}
            \end{aligned}
        \]
        从$\mathcal{I}/S_{1} = \left\{ 1,3 \right\}$中取
        \[
            \begin{array}{l}
                \boldsymbol{a}_{1}^{\mathrm{T}}\boldsymbol{x}^{(2)} = \begin{pmatrix}
                    -1 & -1
                \end{pmatrix}\cdot
                \begin{pmatrix}
                    0 \\ 1
                \end{pmatrix} = -1 =
                \boldsymbol{b}_1 \\
                \boldsymbol{a}_{3}^{\mathrm{T}}\boldsymbol{x}^{(2)} = \begin{pmatrix}
                    0 & 1
                \end{pmatrix}\cdot
                \begin{pmatrix}
                    0 \\ 1
                \end{pmatrix} = -1 \neq
                \boldsymbol{b}_3 = 0
            \end{array}
        \]
        不等式积极约束$i = 1$
        令
        \[
            \boldsymbol{x}^{(2)}=\boldsymbol{x}^{(1)}+\alpha_1\boldsymbol{d}^{(1)}=\begin{pmatrix}0\\1\end{pmatrix},\,S_2 = S_1\cup \{1\} = \{1,2\}
        \]

        \colorbox{cyan!50}{$k = 3$:}

        求解子问题
        \[
            \begin{array}{rl}
                \operatorname*{min}&d_{1}^{2}+d_{2}^{2}-2d_{1}-2d_{2}\\
                \mathrm{s.t.}&-d_{1}-d_2=0\\
                &d_1 = 0
            \end{array}
        \]
        KKT条件
        \[
            \left\{
                \begin{array}{l}
                    2d_1-2+\lambda_1 = 0\\
                    2d_2-2+\lambda_1-\lambda_2 = 0\\
                    d_1+d_2 = 0\\
                    d_1 = 0
                \end{array}
            \right.
        \]
        得最优解
        \[
            \boldsymbol{d}^{(2)}=\begin{pmatrix}0\\0\end{pmatrix},\,\boldsymbol{\lambda}^{(2)}=\begin{pmatrix}2\\0\end{pmatrix}
        \] 
        由$\boldsymbol{d}^{(2)}=0$,且对于$\forall i\in S_{k}\cap \mathcal{I}(\boldsymbol{x}^{(2)}) = \left\{ 1,2 \right\},\lambda_{i}^{(2)}\geqslant 0$算法停止。原问题最优解和最优Lagrange乘子分别为:
        \[
            \boldsymbol{x}^*=\boldsymbol{x}^{(2)}=\begin{pmatrix}0\\1\end{pmatrix},\quad 
            \boldsymbol{\lambda}^* = \begin{pmatrix}
                2\\0\\0
            \end{pmatrix}
        \]
        \[
            f(\boldsymbol{x}^{*}) = -3
        \]
    \end{solution}
\end{example}